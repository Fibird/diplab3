%%%%%%%%%%%%%%%%%%%%%%%%%%%%%%%%%%%%%%%%%
% University Laboratory Report mainly used in programming
% LaTeX Template
% Version 1.0 (12/9/16)
%
% Author: Chaoyang Liu
% E-mail: chaoyanglius@outlook.com 
% (https://chaoyangliu.cc)
%
% License:
% CC BY-NC-SA 3.0 (http://creativecommons.org/licenses/by-nc-sa/3.0/)
%
%%%%%%%%%%%%%%%%%%%%%%%%%%%%%%%%%%%%%%%%%

\documentclass[hyperref,UTF8]{ctexart}
\usepackage{listings} % Required for setting of code block
\usepackage[colorlinks,linkcolor=black]{hyperref}
\usepackage{color}
\usepackage{graphicx} % Required for the inclusion of images
\usepackage{varwidth} % Required for 
\usepackage{float}

\usepackage{graphicx} % Required for the inclusion of images
\usepackage{natbib} % Required to change bibliography style to APA
\usepackage{amsmath} % Required for some math elements 

%\renewcommand{\labelenumi}{\alph{enumi}.} % Make numbering in the enumerate environment by letter rather than number (e.g. section 6)
\CTEXsetup[format={\Large\bfseries}]{section}
%----------------------------------------------------------------------------------------
% SETTING OF CODE BLOCK
%----------------------------------------------------------------------------------------
\lstset{ % 代码高亮
	backgroundcolor=\color{white},   % choose the background color
	basicstyle=\footnotesize\ttfamily,        % size of fonts used for the code
	columns=fullflexible,
	numbers=left,                    % where to put the line-numbers; possible values are (none, left, right)
	numbersep=0.5em,		% how far the line-numbers are from the code
	breaklines=true,                 % automatic line breaking only at whitespace
	captionpos=t,                    % sets the caption-position to bottom
	tabsize=4,
	frame = single,
	framexleftmargin=2em,
	commentstyle=\color{green},    % comment style
	escapeinside={\%*}{*)},          % if you want to add LaTeX within your code
	keywordstyle=\color{blue},       % keyword style
	stringstyle=\color[rgb]{0.58,0,0.82}\ttfamily,     % string literal style
	rulecolor=\color{black},
	% identifierstyle=\color{red},
	language=c++,
	showtabs = false,
	showstringspaces = false,
	showspaces = false,
}

%----------------------------------------------------------------------------------------
% PDF INFORMATION
%----------------------------------------------------------------------------------------
% You need to set it in curly brace
% according to your information
\hypersetup{
	pdftitle={title of your document},
	pdfauthor={your name},
	pdfsubject={subject of your document},
	pdfkeywords={key words},
}

%----------------------------------------------------------------------------------------
%	DOCUMENT INFORMATION
%----------------------------------------------------------------------------------------

\title{频域处理:以傅立叶变换为例} % Title

\author{\kaishu 刘朝洋} % Author name

\date{\today} % Date for the report

\begin{document}

\maketitle % Insert the title, author and date

\begin{center}
\begin{tabular}{l r}
专业班级: & 计算机141 \\ % Date the experiment was performed
学生姓名: & 刘朝洋 \\ % Partner names
指导老师: & 杨龙 % Instructor/supervisor
\end{tabular}
\end{center}

% If you wish to include an abstract, uncomment the lines below
\begin{abstract}
该实验报告中主要使用OpenCV库函数对某个图像进行离散傅立叶变换。另外,为了进一步理解离散傅立叶变换的性质,又分别对几何变换后的图像进行傅里叶变换。通过对比实验结果来验证离散傅立叶变换的性质。
\end{abstract}

\pagestyle{plain}
%----------------------------------------------------------------------------------------
%	SECTION 1
%----------------------------------------------------------------------------------------

\section{实验目的}

\begin{enumerate}

\item 理解傅里叶变换的原理及方法;
\item 学会使用OpenCV对图像进行傅里叶变换;
\item 理解傅里叶变换的性质。

\end{enumerate}
 
%----------------------------------------------------------------------------------------
%	SECTION 2
%----------------------------------------------------------------------------------------

\section{实验内容}

\begin{enumerate}

\item 使用OpenCV库函数对原图像进行傅里叶变换;
\item 将原图像旋转一定角度再进行傅里叶变换,对比变换结果;
\item 将原图像平移一定单位再进行傅里叶变换,对比变换结果;
\item 将原图像缩小一定比例再进行傅里叶变换,对比变换结果;
\item 对图像进行镜像变换再进行傅里叶变换,对比变换结果;

\end{enumerate}

%----------------------------------------------------------------------------------------
%	SECTION 3
%----------------------------------------------------------------------------------------

\section{实验过程}

\subsection{对图像进行傅里叶变换}

下面是对图像进行傅里叶变换的主要步骤:

\begin{enumerate}

\item 扩展原图像已获得DFT处理的最佳大小;

\begin{lstlisting}
// Expand the image to an optimal size
	int r = getOptimalDFTSize(transformedSrc.rows);
	int c = cvGetOptimalDFTSize(transformedSrc.cols);
	copyMakeBorder(transformedSrc, padded, r - transformedSrc.rows, 0, c - transformedSrc.cols, 0, BORDER_CONSTANT);
\end{lstlisting}
\item 将图像转换为复数形式,并将像素的类型转换为\lstinline{float};

\begin{lstlisting}
// Make place for both the complex and the real values
	Mat planes[] = { Mat_<float>(padded), Mat::zeros(padded.size(), CV_32F) };
	Mat complexSrc;
	merge(planes, 2, complexSrc);
\end{lstlisting}

\item 调用OpenCV库函数对扩展后的图像进行傅里叶变换;
\begin{lstlisting}
// Make the Discrete Fourier Transform
	dft(complexSrc, complexSrc);
\end{lstlisting}
\item 将傅里叶变换结果转为模长,得到变换的频谱图;

\begin{lstlisting}
// Transform the real and complex values to magnitude
	split(complexSrc, planes);
	Mat mag(transformedSrc.size(), CV_32F);
	magnitude(planes[0], planes[1], mag);
\end{lstlisting}

\item 为了便于观察,将图像进行对数拉伸和均衡化处理。

\begin{lstlisting}
// Switch to a logarithmic scale
	mag += Scalar::all(1);
	log(mag, mag);
// Normalize
	normalize(mag, mag, 0, 1, CV_MINMAX);
\end{lstlisting}
\item 为了便于分析和处理,将频谱原点移到图像中心;

\begin{lstlisting}
	int cr = mag.rows / 2; int cc = mag.cols / 2;
	Mat tl(mag, Rect(0, 0, cc, cr));
	Mat tr(mag, Rect(cc, 0, cc, cr));
	Mat bl(mag, Rect(0, cr, cc, cr));
	Mat br(mag, Rect(cc, cr, cc, cr));

	Mat temp;
	tl.copyTo(temp);
	br.copyTo(tl);
	temp.copyTo(br);

	tr.copyTo(temp);
	bl.copyTo(tr);
	temp.copyTo(bl);

\end{lstlisting}

\end{enumerate}


\subsection{DFT的时移性质}



可以通过下面的方式使用数学公式:

\begin{equation}
AB^2 = BC^2 + AC^2
\end{equation}

如果要在句中使用数学符号或公式,可以使用$3 \times 3$。不同的数学符号有不同的命令,具体可以在参考该网站:\url{http://meta.math.stackexchange.com/questions/5020/mathjax-basic-tutorial-and-quick-reference}
%----------------------------------------------------------------------------------------
%	SECTION 4
%----------------------------------------------------------------------------------------

\section{结果与结论}

\subsection{内容}

这里主要是陈述实验结果,并根据实验结果得出实验结论。不管实验结果怎样,必须如实地描述。为了更好地呈现实验结果,可以采用图表的形式。

\subsection{插入图表}

我们可以使用下面的方式插入图片:

\begin{figure}[H]
\centering
\includegraphics[width=0.45\textwidth,height=0.25\textwidth]{placeholder} % Include the image placeholder.png
\caption{Figure caption.}
\label{fig:image}
\end{figure}

当然,插入表格的方法也很类似,如下表所示:

\begin{figure}[H]
\centering
\begin{tabular}{|l|c|r|}
\hline
left & center & right	\\
\hline
本列左对齐 & 本列居中 & 本列右对齐	\\
\hline
\end{tabular}
\caption{表格示例}
\end{figure}

%----------------------------------------------------------------------------------------
%	SECTION 5
%----------------------------------------------------------------------------------------

\section{实验总结}

实验总结

%----------------------------------------------------------------------------------------
%	BIBLIOGRAPHY
%----------------------------------------------------------------------------------------

\bibliographystyle{apalike}

\bibliography{sample}

%----------------------------------------------------------------------------------------


\end{document}